%%
%% Jason's preamble.
%%

%
% Need these packages... 
%
\usepackage{latexsym}
\usepackage{amsmath}
\usepackage{amstext}
\usepackage{amsfonts}
\usepackage{amsopn}


\usepackage{ifthen}

%
% Theorem like environments.
%
\newtheorem{theorem}{Theorem}[section]
\newtheorem{lemma}[theorem]{Lemma}
\newtheorem{corollary}[theorem]{Corollary}
\newtheorem{proposition}[theorem]{Proposition}
\newtheorem{definition}{Definition}[section]
%\newtheorem{example}{Example}[section]
\newtheorem{remark}{Remark}[section]

%
% Don't use this, `proof' and `proofby' redefine this env
% to allow you to specify a proof with a different type of 
% enclosing environment for the proof body.
%
\newenvironment{ProofDummyEnv}{}{}

%
% This is an environment that does nothing.  You could use this
% for a proof if you did not want an enclosing environment.
%
\newenvironment{none}{}{}

%
%\begin{proof}[env] proof body \end{proof}
%   
%   env -- the base environment to do the proof.  Can be `itemize',
%          `enumerate', etc.  The default is `quote'
%   proof body -- the actuall proof.
%
\newenvironment{proof}[1][none]{\begin{proofby}[#1]{}}{\end{proofby}}

%
%\begin{proofby}[env]{method} proof body \end{proof}
%
%   env -- the base environment to do the proof.  Can be `itemize',
%          `enumerate', etc.  The default is `quote'
%   method -- could be some text like ``by induction on $n$''
%   proof body -- the actuall proof.
%
\newenvironment{proofby}[2][none]{\vskip\belowdisplayskip\par\noindent{\bf Proof:} #2
\renewenvironment{ProofDummyEnv}{\begin{#1}}{\end{#1}}%
\begin{ProofDummyEnv}}%
{\QED\vskip\belowdisplayskip\end{ProofDummyEnv}}

%
% often used in proofs.
%
\newcommand{\ih}{inductive hypothesis}


%
% BlackBoardBold symbols.
%
\newcommand{\integers}{\mathbb Z}
\newcommand{\reals}{\mathbb R}
\newcommand{\complex}{\mathbb C}
\newcommand{\rationals}{\mathbb Q}
\newcommand{\naturals}{\mathbb N}
\newcommand{\F}{\mathbb F}


%
% this doesn't really work.
%
\newcommand*{\ubar}[1]{\ensuremath{\underline{#1}}}



%
% This adds the cute little box at the end of a proof.
% the proof and proofby environment automatically do this.
%
\newcommand{\QED}{\nopagebreak\hfill $\Box$}

%
% For referencing equations, lemmas, theorems etc.
%
\newcommand{\lref}[1]{Lemma~\ref{#1}}
\newcommand{\eqnref}[1]{Equation~\ref{#1}}
\newcommand{\thmref}[1]{Theorem~\ref{#1}}


%
% these are for text.  some people like to write iff or WLOG, so
% we will provide functions.  Remempler to use them like this:
% ``yadda yadda X \IFF\ Y yadda''
% or
% ``yadda \WLOG, yadda''
%
\newcommand{\IFF}{if and only if}
\newcommand{\WLOG}{without loss of generality}

% isomorphic to: this is =~
\newcommand{\iso}{\cong}

\newcommand{\defeq}{\stackrel{\mathrm{\small def}}{=}}

% derivitives.
\newcommand{\deriv}[2]{\tfrac{d #1 }{d #2}}
\newcommand{\partialderiv}[2]{\tfrac{\partial #1 }{\partial #2}}

%
% probability stuff.
%
\newcommand{\prob}[2][]{\text{\bf Pr}\ifthenelse{\not\equal{}{#1}}{_{#1}}{}\!\left[#2\right]}
\newcommand{\expect}[2][]{\text{\bf E}\ifthenelse{\not\equal{}{#1}}{_{#1}}{}\!\left[#2\right]}
\newcommand{\given}{\,\mid\,}

%
% evaluation macros
%
\newcommand{\inteval}[1]{\Big[#1\Big]}
\newcommand{\evalat}[2]{\left.#1\vphantom{\big|}\right|_{#2}}


%
% random functions of one argument.
%
\newcommand{\ceil}[1]{\left\lceil#1\right\rceil}
\newcommand{\floor}[1]{\left\lfloor#1\right\rfloor}
\newcommand{\abs}[1]{\left| #1 \right|}
\newcommand{\setsize}[1]{\left| #1 \right|}
\newcommand{\norm}[1]{\left\| #1 \right\|}
\newcommand{\pair}[1]{\left\langle #1 \right\rangle}
%
% random functions with no arguments.
%
\newcommand{\divides}{\mid}

%
% Such that is for equations not for text.
%
\newcommand{\suchthat}{\ :\ }


%
% quickly define math abreviations for repeated forms. e.g.
% \mathdef{\PIi}{\pi_i}
%
% which allows you to use $x_\PIi$ instead of $x_{\pi_i}$
%
\newcommand{\mathdef}[2]{\newcommand{#1}{\ensuremath{#2}}}

%
% New math operators.  These work like `gcd' or `log'.
%
\DeclareMathOperator{\lcm}{lcm}
\DeclareMathOperator{\img}{img}
\DeclareMathOperator{\GL}{GL}
\DeclareMathOperator{\argmax}{argmax}
\DeclareMathOperator{\argmin}{argmin}
\DeclareMathOperator{\avg}{avg}
%\DeclareMathOperator{\det}{det}



\newcommand{\super}[1]{^{(#1)}}



%
% the end
%
