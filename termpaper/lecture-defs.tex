
\usepackage{ifthen}
\usepackage{amsmath}
\usepackage{pst-all}
\usepackage{fullpage}
\usepackage{framed}
\usepackage{environ}

%\setlength{\parsep}{3cm}



% fix lengths for displayed math.
\makeatletter
\g@addto@macro\normalsize{\setlength\abovedisplayskip{0pt plus 2pt}}
\g@addto@macro\normalsize{\setlength\belowdisplayskip{0pt plus 2pt}}
\g@addto@macro\normalsize{\setlength\abovedisplayshortskip{0pt plus 2pt}}
\g@addto@macro\normalsize{\setlength\belowdisplayshortskip{0pt plus 2pt}}
\makeatother

% set lengths
\setlength{\parindent}{0ex}
\setlength{\parskip}{2ex plus 0ex minus 1ex}






\newcommand{\forloop}[5][1]%
{%
\setcounter{#2}{#3}%
\ifthenelse{#4}%
	{%
	#5%
	\addtocounter{#2}{#1}%
	\forloop[#1]{#2}{\value{#2}}{#4}{#5}%
	}%
% Else
	{%
	}%
}%



\psset{levelsep=1cm}

%%
%% binary trees
%%
\newcommand{\binnode}[1]{\ifthenelse{\equal{null}{#1}}{\Tn}{\Tcircle{#1}}}
\newcommand{\bintree}[3]{\pstree{\Tcircle{#1}}{{#2}{#3}}}
\newcommand{\lefttree}[2]{\bintree{#1}{#2}{\binnode{null}}}
\newcommand{\righttree}[2]{\bintree{#1}{\binnode{null}}{#2}}
\newcommand{\subtree}[1]{\Ttri{#1}}

%%
%% possibly null subtree 
%%

\DeclareRobustCommand{\tree}[2]{\ifthenelse{\equal{#2}{}}{\Tcircle{#1}}{\pstree{\Tcircle{#1}}{#2}}}

%\DeclareRobustCommand{\rightleaning}[2]{\ifthenelse{\equal{#2}{}}{#1}{#1#2}}

%%%
%%% binomial heap
%%%

%%
%% all trees take an extra argument that attaches a subtree to the root.
%%
\newcommand{\binomialmerge}[3]% binomial heap builder: #1 = smaller-heap; #2 = larger-heap; #3 = more children of root.
{#1{#2#3}}

%%
%%
%%
\newcommand{\BHx}[2]% binomial heap builder 1: #1 = key of root; #2 = more children of root
{\tree{#1}{#2}}

\newcommand{\BHxx}[3]{\binomialmerge{\BHx{#1}}{\BHx{#2}{}}{#3}}
\newcommand{\BHxxxx}[5]{\binomialmerge{\BHxx{#1}{#2}}{\BHxx{#3}{#4}{}}{#5}}
\newcommand{\BHxxxxxxxx}[9]{\binomialmerge{\BHxxxx{#1}{#2}{#3}{#4}}{\BHxxxx{#5}{#6}{#7}{#8}{}}{#9}}

%\newcommand{\BHxx}[3]{\BHx{#1}{{\BHx{#2}{}}{#3}}}
%\newcommand{\BHxxxx}[5]{\BHxx{#1}{#2}{{\BHxx{#3}{#4}{}}{#5}}}
%\newcommand{\BHxxxxxxxx}[9]{\BHxxxx{#1}{#2}{#3}{#4}{{\BHxxxx{#5}{#6}{#7}{#8}{}}{#9}}}


%%
%% linked lists
%%
\newcommand{\nodetemplate}[2]{%
\begin{tabular}{|c|c|}
\hline
#1 & #2 \\ \hline
\end{tabular}}

%%
%% #1 node value
%% #2 node label
%%
\newcommand{\llnode}[2]{\rnode{#2}{\nodetemplate{#1}{\rnode{#2next}{$\bullet$}}}}

%%
%% #1 node value
%% #2 node label
%%
\newcommand{\nullnode}[2]{%
\rnode{#2}{\nodetemplate{#1}{$/$}}%
}

%%
%% #1 node label
%% #2 node label
%%
\newcommand{\link}[2]{\ncline{->}{#1next}{#2}}


\newcommand{\linkednode}[3]{%
\rnode{#2}{\llnode{#1}{\rnode{#2next}{$\bullet$}}}%
\link{#2}{#3}%
}


%%
%% arrays
%%
\newcommand{\placeat}[2]{\nput{u}{#1}{#2}}

\newcommand{\ith}[3][\arabic{arrayfigure}]{\placeat{array#1i#2}{#3}}

\newcommand{\arraycelli}[2]% #1 = array-name; #2 = index
{\begin{pspicture}(0,0)(1,2){\psframe(0,0)(1,1)}\uput[u](0.5,1){#2}\pnode(0.5,.1){array#1i#2}\end{pspicture}}%

%%
%% #1 minimum index
%% #2 maximum index
%% #3 space for each entry
%%
\newcounter{arrayfigure}
\newcounter{arrayfigurei}
\setcounter{arrayfigure}{0}
\newenvironment{arrayfigure}[3]{%
\psset{yunit=4ex,xunit=#3}%
\forloop{arrayfigurei}{#1}{\not\(\value{arrayfigurei} > #2\)}%
{\arraycelli{\arabic{arrayfigure}}{\arabic{arrayfigurei}}}%
}{}




\newcommand{\false}{\equal{1}{2}}
\newcommand{\true}{\equal{1}{1}}

%%
%% notes formatting
%%

\newcommand{\CITE}[1]{{[#1]}}

\newcommand{\ifnocomments}[1]{}
\newcommand{\ifcomments}[1]{}
\newcommand{\minicomment}[2]{}
\newcommand{\comment}[1]{}
\NewEnviron{C}{}

\newlength{\commentwidth}
\setlength{\commentwidth}{\columnwidth}
\addtolength{\commentwidth}{-4ex}


\newcommand{\showcomments}{%
\renewcommand{\ifnocomments}[1]{}
\renewcommand{\ifcomments}[1]{{##1}}
\renewcommand{\minicomment}[2][]{##1{\red $\left[\left[\text{{\it{##2}}}\right]\right]$}}
\renewcommand{\comment}[1]{\begin{C}##1\end{C}}
\RenewEnviron{C}{\red\begin{leftbar}\BODY\end{leftbar}}
}

\newcommand{\hidecomments}{%
\RenewEnviron{C}{}
\renewcommand{\ifnocomments}[1]{{##1}}
\renewcommand{\minicomment}[2][]{}
\renewcommand{\comment}[1]{}
\renewcommand{\ifcomments}[1]{}
}




\showcomments


\newcommand{\noterule}{\hrule}

\newcommand{\notespace}{\vspace{1cm}}
\newcommand{\notesmallspace}{\vspace{.5cm}}

\newcommand{\Xcomment}[1]{}

\newcommand{\h}[1]{\section*{#1}}
\newcommand{\hh}[1]{\subsection*{#1}}
\newcommand{\hhh}[1]{\subsubsection*{#1}}


\newcommand{\notelike}[2][]{{\bf #2\ifthenelse{\equal{#1}{}}{}{ #1}: }}

\newcommand{\newnote}[2]{\newcommand{#1}[1][]{{\bf #2\ifthenelse{\equal{##1}{}}{}{ ##1}: }}}

\newnote{\announce}{Announcements}
\newnote{\claim}{Claim}
\newnote{\defn}{Def}
\newnote{\q}{Question}
\newnote{\oq}{Open Question}
\newnote{\qs}{Questions}
\newnote{\note}{Note}
\newnote{\goal}{Goal}
\newnote{\result}{Result}
\newnote{\idea}{Idea}
\newnote{\recall}{Recall}
\newnote{\soln}{Solution}
\newnote{\obs}{Observation}
\newnote{\ans}{Answer}
\newnote{\challenge}{Challenge}
\newnote{\example}{Example}
%\newnote{\alg}{Algorithm}
\newnote{\fact}{Fact}
\newnote{\reading}{Reading}
\newnote{\miniproof}{Proof}
\newnote{\story}{Story}
\newnote{\thm}{Theorem}
\newnote{\cor}{Corollary}
\newnote{\lem}{Lemma}
\newnote{\prop}{Prop}
\newnote{\examples}{Examples}
\newnote{\pf}{Proof}
\newnote{\pfsketch}{Proof Sketch}
%\newnote{\solution}{Solution}
\newnote{\assume}{Assumption}
\newnote{\conc}{Conclusion}
\newnote{\conj}{Conjecture}
\newnote{\step}{Step}


\newcommand{\then}{$\Rightarrow$}
\renewcommand{\QED}{{\bf QED}}
\newcommand{\contradiction}{$\rightarrow\leftarrow$}






\newcommand{\hr}{\rule{\columnwidth}{.5pt}}

\newcommand{\thecoursenumber}{EECS~\#\#\#}
\newcommand{\thecoursename}{Course Name}
\newcommand{\thelecture}{\#}
\newcommand{\thetopic}{Topic}
\newcommand{\thesubtopics}{Subtopics}
\newcommand{\coursename}[1]{\renewcommand{\thecoursename}{#1}}
\newcommand{\coursenumber}[1]{\renewcommand{\thecoursenumber}{#1}}
\newcommand{\lecturenumber}[1]{\renewcommand{\thelecture}{#1}}
\newcommand{\lecturetopic}[1]{\renewcommand{\thetopic}{#1}}
\newcommand{\lecturesubtopics}[1]{\renewcommand{\thesubtopics}{#1}}

\newcommand{\makelecturetitle}{%
\author{}
\date{}
\title{\noindent\fbox{\parbox{6.4in}{
\bf \large \thecoursenumber: \thecoursename
        \hfill Lecture \thelecture\\
\thetopic \hfill \rm\small\thesubtopics}}}
\maketitle}




\newcommand{\newzerocounter}[2][0]{\newcounter{#2}\setcounter{#2}{#1}}


\newenvironment{notelist}
{\begin{list}{}{\setlength{\parsep}{\parskip}\setlength{\topsep}{0in}\setlength{\rightmargin}{0in}\setlength{\itemsep}{0in}}}
{\end{list}}

\newzerocounter{notesnumber}

\newcounter{notesimbalance}

\newif\ifsuspendimbalance
\suspendimbalancefalse


\newcommand{\printincrement}[2]{{#1{#2}}\addtocounter{#2}{1}}

%
% plain
%
\newcommand{\itemorig}{\item}
\newcommand{\itemnote}{\fiximbalance\suspendimbalancefalse\itemorig}
\newcommand{\itemout}{\itemnote\hspace{-.5\labelwidth}\hspace{-.5\labelsep}}

%
% bullets and impications
%
\newcommand{\itembull}{\itemnote[$\bullet$]}
\newcommand{\itemimpl}{\itemnote[$\Rightarrow$]}

%
% numbers.
%
\newcommand{\itemnumber}{\itemnote[\notecounter{\compose{\periodafter}{\arabic}}]}
\newcommand{\itemroman}{\itemnote[\notecounter{\compose{\parens}{\roman}}]}
\newcommand{\itemRoman}{\itemnote[\notecounter{\compose{\periodafter}{\Roman}}]}
\newcommand{\itemalpha}{\itemnote[\notecounter{\compose{\parens}{\alph}}]}

%
% 
%

\newcommand{\notecounter}[1]{\printincrement{#1}{notes\arabic{notesnumber}-\roman{noteslevel\arabic{notesnumber}}}}

\newcommand{\notecounterreset}[1][1]{\setcounter{notes\arabic{notesnumber}-\roman{noteslevel\arabic{notesnumber}}}{#1}}


\newcommand{\fiximbalance}{\ifsuspendimbalance\relax\else\recursivefiximbalance\fi}
\newcommand{\recursivefiximbalance}{\ifnum\value{notesimbalance}=0\relax\else
\endnotelist\addtocounter{notesimbalance}{-1}\recursivefiximbalance
\fi}

\newcommand{\closenote}{\addtocounter{notesimbalance}{\value{noteslevel\arabic{notesnumber}}}\setcounter{noteslevel\arabic{notesnumber}}{0}\fiximbalance}

\newcommand{\notein}{\fiximbalance\addtocounter{noteslevel\arabic{notesnumber}}{1}\notelist}
\newcommand{\noteout}{\fiximbalance\endnotelist\addtocounter{noteslevel\arabic{notesnumber}}{-1}}
\newcommand{\notequickin}{\fiximbalance\addtocounter{notesimbalance}{1}\suspendimbalancetrue\notelist}


\newenvironment{notes}
{\newzerocounter{noteslevel\arabic{notesnumber}}%
\setcounter{notesimbalance}{0}%
\newzerocounter[1]{notes\arabic{notesnumber}-i}%
\newzerocounter[1]{notes\arabic{notesnumber}-ii}%
\newzerocounter[1]{notes\arabic{notesnumber}-iii}%
\newzerocounter[1]{notes\arabic{notesnumber}-iv}%
\let\itemorig\item%
\def\item{\itemnote}%
\def\+{\notein}%
\def\-{\noteout}%
\def\>{\notequickin}%
\def\o{\itembull}%
\def\so{\itemimpl}%
\def\LEQ{\item[$\leq$]}
\def\GEQ{\item[$\geq$]}
\def\eq{\item[$=$]}%
\def\EQ{\item[$=$]}%
\def\i{\itemroman}%
\def\a{\itemalpha}%
\def\n{\itemnumber}%
\def\I{\itemRoman}
\def\<{\itemout}%
\def\tab{\hspace{3ex}}%
\def\reset{\notecounterreset}%
\def\`{\item\hfill}%
}
{\closenote\addtocounter{notesnumber}{1}}







\newcounter{notablelevel}
\newcounter{notablei}
\newcounter{notableii}
\newcounter{notableiii}
\newcounter{notableiv}

\newcommand{\nota}[1]{#1\'}
\newcommand{\makenota}[2]{\newcommand{#1}{\nota{#2}}}
\newcommand{\noter}[1]{{#1{notable\roman{notablelevel}}}\addtocounter{notable\roman{notablelevel}}{1}}
\newcommand{\noterclear}[1][1]{\setcounter{notable\roman{notablelevel}}{#1}}
\newcommand{\notableintmp}{\+}
\newcommand{\notableouttmp}{\-}
\newcommand{\notablein}{\notableintmp\addtocounter{notablelevel}{1} a}
\newcommand{\notableout}{\notableouttmp\addtocounter{notablelevel}{-1} b}
\makenota{\notabull}{$\bullet$}
\makenota{\notaimpl}{$\Rightarrow$}
\makenota{\notanumb}{\noter{\compose\periodafter\arabic}}
\makenota{\notaroma}{\noter{\compose\parens\roman}}
\makenota{\notaRoma}{\noter{\compose\periodafter\Roman}}
\makenota{\notalpha}{\noter{\compose\parens\alph}}



\newenvironment{notable}%
{\def\o{\notabull}%
\def\so{\notaimpl}%
\def\a{\notalpha}%
\def\i{\notaroma}%
\def\n{\notanumb}%
\def\I{\notaRoma}%
\setcounter{notablelevel}{1}%
\setcounter{notablei}{1}%
\setcounter{notableii}{1}%
\setcounter{notableiii}{1}%
\linespread{1.6}%
\begin{tabbing}%
\let\notableintmp\+%
\let\notableouttmp\-%
\def\+{\notablein}%
\def\-{\notableout}%
XXX\=XXX\=XXX\=XXX\=XXX\=XXX\=\kill%
}%
{\end{tabbing}}













\newlength{\quickinsep}
\newlength{\quicklabelsep}
\newlength{\quicksep}
\setlength{\quicksep}{.1cm}
\setlength{\quickinsep}{5ex}
\setlength{\quicklabelsep}{4ex}
\newcommand{\quickinbox}{\makebox[\quickinsep]{}}
\newcommand{\quickin}[1][1]{\addtocounter{quicklevel}{#1}}
\newcommand{\quickout}[1][1]{\addtocounter{quicklevel}{-#1}}
\newcommand{\quickallout}{\setcounter{quicklevel}{0}}
\newcommand{\quick}[1][]{\\[\quicksep]\makebox[0in]{\hspace{-1cm}\arabic{quicklevel}}\forloop{quicktmp}{0}{\value{quicktmp} < \value{quicklevel}}{\>}\makebox[0in]{\hspace{-\quicklabelsep}#1}}
\newcommand{\newquickie}[2]{\newcommand{#1}{\quick[#2]}}
\newquickie{\bull}{$\bullet$}
\newquickie{\impl}{$\Rightarrow$}
\newcommand{\quicklevel}[1][\arabic]{{#1{quickcount\roman{quicklevel}}\addtocounter{quickcount\roman{quicklevel}}{1}}}
\newcommand{\quicknumb}[1][\compose\periodafter\arabic]{\quick[{\quicklevel[#1]}]}
\newcommand{\newquicknumb}[2]{\newcommand{#1}{\quicknumb[#2]}}
\newquicknumb{\quickalph}{\compose\parens\alph}
\newquicknumb{\quickroma}{\compose\parens\roman}
\newquicknumb{\quickRoma}{\compose\periodafter\Roman}
\newcommand{\num}{\bull}
\newcommand{\periodafter}[1]{#1.}
\newcommand{\parens}[1]{(#1)}

\newcommand{\compose}[3]{#1{#2{#3}}}


\newcommand{\renumb}{\setcounter{quickcount\roman{quicklevel}}{0}}

\newcounter{quicktmp}
\newcounter{quicklevel}
\newcounter{quickcounti}
\newcounter{quickcountii}
\newcounter{quickcountiii}
\newcounter{quickcountiv}


\newenvironment{quickie}
{%
\newcommand{\In}{\quickin}%
\newcommand{\Out}{\quickout}%
\newcommand{\Q}{\quick}%
\renewcommand{\a}{\quickalph}%
\renewcommand{\i}{\quickroma}%
\newcommand{\I}{\quickRoma}%
\newcommand{\n}{\quicknumb}%
\renewcommand{\o}{\bull}%
\begin{tabbing}%
\setcounter{quicklevel}{0}%
\setcounter{quickcounti}{1}%
\setcounter{quickcountii}{1}%
\setcounter{quickcountiii}{1}%
\setcounter{quickcountiv}{1}%
\quickinbox\=\quickinbox\=\quickinbox\=\quickinbox\=\quickinbox\=\kill%
}
{\end{tabbing}}



\newcounter{labelitcounter}
\setcounter{labelitcounter}{0}

%\newcommand{\labelitdecode}[1]{4,4}

\newcommand{\labelitsetwhere}[1]{%
\ifthenelse{\equal{ur}{#1}}{\renewcommand{\labelitwhere}{4,4}}{}%
\ifthenelse{\equal{dr}{#1}}{\renewcommand{\labelitwhere}{4,-3}}{}%
\ifthenelse{\equal{ul}{#1}}{\renewcommand{\labelitwhere}{-4,4}}{}%
\ifthenelse{\equal{dl}{#1}}{\renewcommand{\labelitwhere}{-4,-3}}{}%
}

\newcommand{\labelitwhere}{4,4}

\newcommand{\labelit}[3][ur]{\labelitsetwhere{#1}{\psset{unit=1ex}\addtocounter{labelitcounter}{1}\rnode{labelitA\roman{labelitcounter}}{#2}\rput(\labelitwhere){\rnode{labelitB\roman{labelitcounter}}{#3}}\ncline{->}{labelitB\roman{labelitcounter}}{labelitA\roman{labelitcounter}}}}
